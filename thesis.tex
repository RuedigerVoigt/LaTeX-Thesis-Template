% !TEX encoding = UTF-8 Unicode
\documentclass[11pt, oneside]{book}
% * The documentclass book is a good choice for thesis. For
%   a short text like a course homework the documentclass
%   article would be more suitable.
% * The fontsize specified here is the base text font size 
%   for the document. Headings, footnotes, and other text
%   are automatically scaled relative to that. Typically the
%   documentclass sets a maximum.
% * The option oneside avoids empty pages in drafts, but 
%   should be removed in production: Readers expect chapters
%   to start at the right side. / ... => Do NOT forget
%   to turn on the options inner/outer for geometry after
%   removing oneside!
% 

% check if the editor's file encoding is set accordingly:
\usepackage[utf8]{inputenc}
\usepackage[T1]{fontenc}

% * All languages here are available, but the one named last
%   will be loaded as default for this document.
% * So if you want to use another of these languages as default,
%   you have to switch the order. As a result headers et cetera 
%   will change.
% * ngerman is German with new spelling.
\usepackage[french,ngerman,english]{babel}

% Classic LaTeX would use the \hyphenation{} command.
% Since babel 3.9a* hyphenation can be specified 
% language specific.
% * = introduced 2017, but might require an update
%     depending on your distribution
\babelhyphenation[english]{}
\babelhyphenation[ngerman]{}


% %%%%%%%%%%%%%%%%%%%%%%%%%%%%%%%%%%%%%%%%%%%%%%%%%%%%%%%%%%%
% START: CHANGE BEFORE PRODUCTION
% %%%%%%%%%%%%%%%%%%%%%%%%%%%%%%%%%%%%%%%%%%%%%%%%%%%%%%%%%%%

% Change left/right to inner/outer once you turn of
% the oneside option!
\usepackage[a4paper,
left = 2.5cm,
right = 6cm, % more space for handwritten notes
top=2.5cm,
bottom=2.5cm]{geometry}

% more options to print out time
% BEWARE: it is datetime2 not datetime, which is outdated.
% provides the \DTMnow command
\usepackage[showzone=False]{datetime2}

\usepackage{background}
\backgroundsetup{
	position=current page.west,
	angle=90,
	vshift=-10mm,
	color=gray,
	opacity=1,
	scale=1,
	contents={\emph{DRAFT \DTMnow}}
}

% showidx has to be turned off, too.
% It had to be loaded after makeidx, so see below.

% %%%%%%%%%%%%%%%%%%%%%%%%%%%%%%%%%%%%%%%%%%%%%%%%%%%%%%%%%%%
% END: CHANGE BEFORE PRODUCTION
% %%%%%%%%%%%%%%%%%%%%%%%%%%%%%%%%%%%%%%%%%%%%%%%%%%%%%%%%%%%


\usepackage{parskip}


% Load Options for line spacing. The change in 
% spacing is activated later in the document itself.
% For example with \onehalfspacing.
\usepackage{setspace}

% Select a font:
% * That should be a serif type as it is a long text 
%   and serif is better to read than non-serif.
%\usepackage{lmodern}
\usepackage{palatino}


% MATH
% * You always need amsmath and amssymb.
\usepackage{amsmath}
\usepackage{amssymb}
\usepackage{cancel} % allows to strike out terms
\usepackage{mathtools}

% TABLES
\usepackage{booktabs} % better looking tables
\usepackage{longtable} % multi-page tables
\usepackage{multirow} % join rows
\usepackage{array}
\usepackage{dcolumn} % align column on a delimiter
\usepackage{tabularx}

% GRAPHICS
\usepackage{graphicx} % Provides \includegraphics{}
\usepackage{tikz}

\usepackage{microtype}


% Format the caption 
\usepackage[
labelfont=bf,
format=hang,
labelsep=colon,
justification=justified,
position=bottom,
figurename=Fig.,
tablename=Tab.]{caption}



% The ulem package makes it possible to underline 
% or strike out text. The option `normalem' avoids 
% redefinition of \emph{}.
\usepackage[normalem]{ulem}

% csquotes introduces the \enquote{} command, which
% sets quotation marks matching the current language.
\usepackage[autostyle=true]{csquotes}

% Changing the footnote style:
\usepackage[marginal]{footmisc}


\usepackage{float}


% Used in this template to generate sample texts.
\usepackage{blindtext}


\usepackage[backend = biber, % because we use utf8 encoding
style=authoryear, % cite with last name author(s) and year
sorting = nyt % how to sort the bibliography
]{biblatex}
\addbibresource{example-literature.bib}


% INDEX
\usepackage{makeidx}
\makeindex

% To show index entries on the page margin use the 
% showidx package. In a draft this is helpful, but 
% before release the package should be removed.
\usepackage{showidx}


% In case we want to show computer code, listings can
% format code in multiple languages.
\usepackage{listings}


% * Hyperrefs allows internal and external links.
% * Once loaded all entries in the table of contents,
%   footnotes, references, ... become hyperlinks in PDF. 
% * Also it has many options to set parameters of 
%   a PDF file.
% * Most of the time it has to be the last package
%   loaded in the preamble.
\usepackage{hyperref}
\hypersetup{
colorlinks = false, 
hidelinks, % no boxes around links
bookmarksopen = true,
pdftitle = {Thesis Example},
pdfauthor = {Rüdiger Voigt}, 
pdfsubject = {doctoral thesis}
}


% The command \today is changed to the current date 
% every time the document gets recompiled.
% A good way to keep track of the version.
\title{A Thesis Template written in \LaTeX}
\author{\foreignlanguage{ngerman}{Rüdiger Voigt, M.A.}}
\date{DRAFT VERSION \today}

\begin{document}


% !TEX encoding = UTF-8 Unicode
% 
% The titlepage-environment is special. It shows no page
% number and the counter is set in a way, that the next 
% page starts as 1.
\begin{titlepage}
% Normally \meaketitle inserts a pagebreak after the title.
% This makes sense, but in the draft phase I want avoid that.
% So instead of just witing \maketitle, I suppress that behavior
% by using the following line:
{\let\newpage\relax\maketitle}
\ \\
\vfill % fill vertical space and push the paragraph below 
% down the page
%
\ \\
% who wrote this thesis? ...
\foreignlanguage{ngerman}{Rüdiger Voigt, M.A.}\\
Student-ID\\
Address\\
Mail / Phone\\
\url{https://www.ruediger-voigt.eu/}\\
\ \\
% Who is the advisor? ...
Advisor: \foreignlanguage{ngerman}{Professor Dr. X}\\
Chair of \dots\\
University of Cologne\\
\end{titlepage}

% include instead of input inserts a pagebreak.
% !TEX root = thesis.tex
% !TEX encoding = UTF-8 Unicode
\thispagestyle{empty}
\begin{center}
{\Large Statement}
\end{center}
\ \\
Some universities / departments require a written statement, that everything is the student's own work. For the exact wording consult the corresponding statutes. There might even be rules for the exact place in the document.\\
\ \\
\today\ in Cologne\hfill Signature and Name\\
% The backslash followed by a space after the command \today
% forces an empty space between the date filled in and
% the words afterwards.
% Putting two or even more spaces behind the command
% does NOT produce a space in the output.
\frontmatter
\tableofcontents
\listoffigures
\listoftables

% \mainmatter switches to arabic numbers for pages.
% It also restart the page counter so content starts
% at page 1.
\mainmatter
% change linespacing via the linespace package:
\onehalfspacing 
% contents divided into multiple files
% !TEX root = thesis.tex\section{Citations}
% !TEX encoding = UTF-8 Unicode

\chapter{Introduction}\label{chapterIntroduction}

\cite[635-637]{Beck1995}

\cite[430]{Putnam1988}

Fullcite-Command:

\fullcite{Putnam1988}
% !TEX encoding = UTF-8 Unicode
\chapter{Introduction}\label{chapterIntroduction}

The introduction\index{introduction} is important to stir up reader's curiosity\index{curiosity} and to explain why this is an important topic.\\
\ \\
\ \\
\begin{table}[h]
\begin{center}
\begin{tabular}{rl} % defines the columns 
% and their orientation:
% r = right
% l = left
% c = centered
\toprule % needs the booktabs package
Variable & Value\\
\midrule % needs the booktabs package
a & 1\\
bb & 22\\
ccc & 333\\
\bottomrule % needs the booktabs package
\end{tabular}
\label{tabMeine}
\caption{first table}
\end{center}
\end{table}

\blindtext




\section{Itemize}

\begin{itemize}
	\item first item
	\item second item
	\item third item
\end{itemize}
% !TEX root = thesis.tex
% !TEX encoding = UTF-8 Unicode
\chapter{Analysis}\label{chapterAnalysis}






\section{Images}

% Option p and no other
% => that means: put it on a page with other figures
% Alternatives would be h (here), t (top) and b (bottom) or any
% combination of those like [htbp].
\begin{figure}[p]
\centering
\includegraphics{img/Dala}
\label{img:cat}
\caption{My cat in 2018}
\end{figure}

% creates A LOT of text between those images
% but both have the only placement parameter p
\blindmathtrue\Blindtext\Blindtext




\begin{figure}[p]
\centering
% Scale the image so it takes up 50% of the width of the text body.
% Height is scaled accordingly.
\includegraphics[width=0.5\textwidth]{img/toad}
\label{img:toad}
\caption{A toad found in the garden.}
\end{figure}

% !TEX root = thesis.tex
% !TEX encoding = UTF-8 Unicode
\chapter{Conclusion}\label{chapterConclusion}

\Blindtext

% !TEX root = thesis.tex\section{Citations}
% !TEX encoding = UTF-8 Unicode

\chapter{TikZ}

\begin{figure}[htb]
	\center
	\includegraphics{figures/tikz-example/arrows.pdf}
	\label{tikz-example}
	\caption{Example for TikZ}
\end{figure}
% !TEX root = thesis.tex\section{Citations}
% !TEX encoding = UTF-8 Unicode

\chapter{Table Examples}

\begin{table}[h]
\begin{center}
\begin{tabular}{rl} % defines the columns 
% and their orientation:
% r = right
% l = left
% c = centered
\toprule % needs the booktabs package
Variable & Value\\
\midrule % needs the booktabs package
a & 1\\
bb & 22\\
ccc & 333\\
\bottomrule % needs the booktabs package
\end{tabular}
\label{tabMeine}
\caption{first table}
\end{center}
\end{table}

\appendix
% Chapters (...) get letters instead of numbers.
\chapter{Proof of the Main Result}

\backmatter
\printindex

% print the bibliography here
% optional: add it to the table of contents
% optional: customize the title
\printbibliography[heading=bibintoc,title={Bibliography}]

\end{document}
